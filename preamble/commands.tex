% commands.tex

% Fbox padding
\renewcommand{\fboxsep}{4pt}

% Row operation arrow
\newcommand\rowop[1]{\scriptstyle\smash{\xrightarrow[\vphantom{#1}]{\mkern-4mu#1\mkern-4mu}}}

% Convert list of row operations to matrix
\DeclareDocumentCommand\converttorows{>{\SplitList{,}}m}
  {\ProcessList{#1}{\converttorow}}

\NewDocumentCommand{\converttorow}{m}
  {\ifthenelse{\isempty{#1}}{}{\rowop{#1}}\\}

\DeclareDocumentCommand \rowops{m}
{
  \;
  \begin{matrix}
    \converttorows{#1}
  \end{matrix}
  \;
}

% Augmented matrix tweak (removes left indent)
\makeatletter
\renewcommand*\env@matrix[1][*\c@MaxMatrixCols c]{%
  \hskip -\arraycolsep
  \let\@ifnextchar\new@ifnextchar
  \array{#1}}
\makeatother

% Subscripted bmatrix
\newenvironment{sbmatrix}[1]
  {\def\mysubscript{#1}\mathop\bgroup\begin{bmatrix}}
  {\end{bmatrix}\egroup_{\textstyle\mathstrut\mysubscript}}

% Define subitem counter
\newcounter{subitemcounter}

% Save the original \item command
\let\originalitem\item

% Patch \item to reset subitem counter (but only if not called by \subitem)
\renewcommand{\item}{%
  \ifboolexpr{ not test {\ifcsdef{subitemactive}} }%
    {\setcounter{subitemcounter}{0}}%
    {}%
  \originalitem
}

% Define \subitem (preserves subitem counter)
\renewcommand{\subitem}{%
  \csdef{subitemactive}{}% Mark that we're in a subitem
  \stepcounter{subitemcounter}%
  \originalitem[\textnormal{(\alph{subitemcounter})}]%
  \csundef{subitemactive}% Unmark
}
